\subsection{Motivation}

The paradigmatic problem of Fast Multipole Methods (\textbf{\gls{FMM}})
\footnote{The first usage of a technical term or abbreviation listed in the
glossary is highlighted throughout the text for ease of reference.} is the so
called $N$-Body problem. This classic problem refers to the calculation of the pairwise
interactions between $N$ particles over a potentially long-range, for example in
gravitational or electrostatic systems. The straightforward calculation can be
written in the form of the following sum,

\begin{equation}
\Phi(x_j) = \sum_{i=1}^N w_i K(x_i, x_j)
\label{eq:n_body_problem}
\end{equation}

Where $i, j \in [1,N]$ and $K(x, y)$ is called the Green's function, or equivalently
a `kernel function', where one is generally concerned with coordinates of particles in an
$n=2 \> \> \text{or} \> \> 3$ dimensional Hilbert space taking  $\ x_i \in \mathbb{R}^n$.
Additionally, each summand is weighted by $w_i$. For solving for electrostatic
potential in three dimensions, which is used as the model problem throughout this
thesis, this goes to,

\begin{equation}
\Phi(x_j) = \sum_{i=1}^N q_iK(x_i, x_j)
\label{eq:electrostatic_paradigm}
\end{equation}

where $q_i$ refers to a charge density with the kernel function,

\begin{equation}
    K(x, y) = \frac{1}{4\pi \epsilon_0}\frac{1}{| x - y |}
\label{eq:laplace_kernel}
\end{equation}

the constant $\epsilon_0$ is the permittivity of free space. It's easy to see
how a naive direct application of this equation over $N$ particles
results in an algorithm of $O(N^2)$ complexity, therefore it's only practicable
for systems of moderate size, whereas realistic systems, one may be interested in
interactions involving $10^{6}$ to $10^8$ particles.

This chapter introduces the analytic, \gls{FMM}, the kernel-independent
version, which is the main focus of this thesis, is presented later.
Though substantially different in implementation, the analytic FMM will provide the opportunity to exposit many of the key ideas
behind all FMM-based algorithms, and provides a good starting point for understanding
and developing upon these algorithms. First presented by Greengard \cite{Greengard:1987:Yale},
the analytic \gls{FMM} represented a sea change for $N$-Body simulation. By
trading off computations for error, it manages to achieve an asymptotic complexity
 of just $O(N)$. Additionally, it comes equipped with rigorous error bounds,
making fast and accurate massive $N$-Body simulations feasible on available
computing hardware. It's success has been such that it is regarded as one of
the key developments in numerical algorithms in the twentieth century \cite{Cipra:2000:SN}.

The original analytic FMM solves the electrostatic problem
in two and three dimensions, this is equivalently known as the Poisson problem,
represented by the differential equation,

\begin{equation}
    \nabla^2 \phi =f
\label{eq:poisson}
\end{equation}

Where $\phi$ is some scalar potential to be determined, and $f$ is a scalar source
term which is usually known. For electrostatics the corresponding formulation
can be derived from Gauss' law as \cite{Griffiths:2017:CUP},

\begin{equation}
  \nabla^2 \phi = - \frac{q}{\epsilon_0}
\label{eq:electrostatic_poisson}
\end{equation}

where $\phi$ is the electrostatic potential, $q$ is the charge density and
$\epsilon_0$ is the permittivity of free space. It can therefore be seen that
the \gls{FMM} is actually solving the Poisson problem by reformulating it as an
integral equation. The ubiquity of problems of the form (\ref{eq:n_body_problem})
in computational science has lead to diverse application of the FMM. For example,
in the modeling the electrostatic interactions of charged particles in complex
biological molecules at biologically relevant length scales \cite{Board:1992:CPL}.
The extension of FMMs to Helmholtz equations \cite{Rokhlin:1990:JCP}, has lead
to even more applications, such as in seismic and acoustic scattering
\cite{Hwu:2011:MKP}. Though as the focus of this thesis is on solving the Laplace
model problem for electrostatics, this is mentioned only for completeness.

The key insight that leads to the \gls{FMM}'s asymptotic complexity is the idea
that if the field created by a distribution of charge (or mass) density is approximated
to be relatively smooth in the \textbf{\gls{far-field}}, then it should be possible to apply
some form of compression for the evaluation of contribution to local potentials due
to particles in the \gls{far-field}. The FMM performs this compression by encoding
the field contributions of particles in the \gls{far-field} using a multipole
expansion.

For simple kernel functions and charge distributions, such as the model problem
of this thesis, one can easily derive the expression for this multipole expansion
by finding an series expansion of the system's Green's function.
In order to generalise the discussion, an arbitrary continuous
distribution of charge is considered as shown in figure \ref{fig:1_1_continuous_charge_distribution},
for which the potential is evaluated at some other evaluation point outside of
the distribution. This can be written as follows,

\begin{flalign}
    \Phi(\mathbf{r}) = \frac{1}{4 \pi \epsilon_0} \int \frac{1}{d}\rho(\mathbf{r}')d\tau'
    \label{eq:1_1_continuous_integral_formulation}
\end{flalign}

where $\rho(\mathbf{r}')$ is a charge density, and the other symbols take their
meanings from figure (\ref{fig:1_1_continuous_charge_distribution}).


\begin{figure}[!h]
    \centering
    {\includegraphics[width=0.45\textwidth]{introduction/continuous_charge.png}}
  \vspace{0pt}
  \caption{An arbitrary charge distribution, with an orange point to mark a point
  where the potential is being evaluated. Here, $\mathbf{r}$ is the the vector
  between the centre of the multipole expansion and the evaluation point, $\mathbf{r}'$
  is the vector between the centre of expansion and a given volume element $d\tau'$, and
  $d$ is a vector between the volume element $d\tau'$ and the evaluation point.}
  \label{fig:1_1_continuous_charge_distribution}
\end{figure}

From law of the cosines,

\begin{flalign}
    d^2 &= r^2 + (r')^2 - 2rr'\cos \alpha = r^2 \left [ 1 + \left ( \frac{r'}{r} \right)^2 - 2 \left (\frac{r'}{r} \right)\cos \alpha \right]\\
    d &= r \sqrt{1+\epsilon}
    \label{eq:1_1_law_of_cosines}
\end{flalign}

Where,

\begin{flalign}
    \epsilon \equiv \left ( \frac{r'}{r} \right) \left (\frac{r'}{r} - 2 \cos \alpha \right)
\end{flalign}

As $\epsilon$ is small far away from charge distribution one can expand $1/d$ binomially,

\begin{flalign}
    \frac{1}{d} &= \frac{1}{r}(1+\epsilon)^{-1/2} = \frac{1}{r}\left (1 - \frac{1}{2}\epsilon + \frac{3}{8}\epsilon^2 - ... \right) \\
    \frac{1}{d} &= \frac{1}{r} \sum_{n=0}^{\infty} \left(\frac{r'}{r} \right)^n P_n(\cos \alpha)
\end{flalign}

Using this, the exact multipole expansion for this charge distribution is,

\begin{flalign}
    \Phi(\mathbf{r}) = \frac{1}{4 \pi \epsilon_0}\sum_{n=0}^{\infty}\frac{1}{r^{n+1}}\int (r')^nP_n(\cos \alpha)\rho(\mathbf{r'}) d \tau'
\end{flalign}

If instead we consider composed of $N$ charges $q_i$ this goes to,

\begin{flalign}
    \Phi(\mathbf{r}) = \frac{1}{4 \pi \epsilon_0}\sum_{i=1}^N\sum_{n=0}^{\infty}\frac{(r')^n q_i}{r^{n+1}}P_n(\cos \alpha)
\end{flalign}

Using the addition theorem for Legendre polynomials \cite{Greengard:1987:Yale},

\begin{flalign}
    P_n(\cos \gamma) = \sum_{m=-n}^n Y_n^{-m}(\alpha, \beta) Y_n^m(\theta, \phi)
\end{flalign}

Where we have written the Legendre polynomial in terms of spherical harmonics,
where $(r, \theta, \phi)$ and $(\rho, \alpha, \beta)$ define two spherical coordinates,
and $\gamma$ is the angle subtended between them. Therefore, the multipole expansion goes to,

\begin{flalign}
    \Phi(\mathbf{r}) &= \frac{1}{4 \pi \epsilon_0}\sum_{i=1}^N\sum_{n=0}^{\infty}\frac{(r')^n q_i}{r^{n+1}}P_n(\cos \alpha)\\
    & = \frac{1}{4 \pi \epsilon_0}\sum_{i=1}^N\sum_{n=0}^{\infty}\sum_{m=-n}^n \frac{(r')^n q_i Y_n^{-m}(\alpha_i, \beta_i) }{r^{n+1}}Y_n^m(\theta, \phi)\\
    & = \sum_{i=1}^N\sum_{n=0}^{\infty}\sum_{m=-n}^n\frac{M_n^m}{r^{n+1}} \cdot Y_n^m(\theta, \phi)
\end{flalign}

This is an exact expansion, and it converges for $\frac{r'}{r} < 1$. This convergence
condition means estimating of the potential at a given evaluation point
calculated using the multipole expansion is only possible in the far-field, the
boundary of which is often tuned empirically for different systems as it's user
defined. If instead the expansion is taken with centre at the evaluation point,
one can rewrite as the multipole expansion as a `local' expansion,

\begin{flalign}
    \Phi(\mathbf{r}) & = \frac{1}{4 \pi \epsilon_0}\sum_{i=1}^N\sum_{n=0}^{\infty}\sum_{m=-n}^n \frac{(r)^n q_i Y_n^{-m}(\alpha_i, \beta_i) }{r'^{n+1}}Y_n^m(\theta, \phi)\\
    & = \frac{1}{4 \pi \epsilon_0} \sum_{i=1}^N \sum_{n=0}^{\infty} \sum_{m=-j}^n L_m^n \cdot  Y_m^n(\theta, \phi) \cdot r^j
\end{flalign}

which converges when $\frac{r}{r'} < 1$. The region of convergence for both types
of expansions are shown in figure (\ref{fig:1_1_multipole_local_expansions}).

The key point to note is that the multipole and local expansions are exact, and
can be truncated as required to ensure that the asymptotic complexity of evaluating
a multipole or local expansion at an evaluation point is bounded by $O(N)$.

\begin{figure}[!h]
    \centering
    {\includegraphics[width=0.33\textwidth]{introduction/multipole_expansion.png}}
    \hfill
  {\includegraphics[width=0.4\textwidth]{introduction/local_expansion.png}}
  \vspace{0pt}
  \caption{(A) A multipole expansion centered on charge distribution. (B) A local
  expansion, centered around a point of evaluation. Regions in which the expansions
  converge are shaded in grey. For the multipole expansion this is the entire domain
  outside of the region for $r>r'$, and for the local expansion this is the region
  for which $r < r'$}

  \label{fig:1_1_multipole_local_expansions}
\end{figure}

\hspace{10pt}

\subsection{Algorithm structure and analysis}

The convergence condition of the multipole expansion prohibits the compression
of charges from particles in the \textbf{\gls{near-field}}. Therefore the FMM
makes use of a tree structure in a recursive algorithm, known as an Octree in
three dimensions and a Quadtree in two dimensions. This structure hierarchically
partitions space such that each level, $l$, of the tree is equally partioned into
$(2^n)^l$ boxes over the domain of the tree, where $n$ is the dimension,
i.e. $n=3$ in three dimensions. If one were to simply traverse the tree from
the coarsest level to the finest, or `leaf', level and find the multipole expansion of source
particles in each box of each level, one could then evaluate these multipole expansions
at each particle to solve the $N$-Body problem. As there are $O(\log(N))$
boxes in the tree, and $N$ particles this results in a $O(N\log(N))$ asymptotic
complexity.

However the FMM reduces computational complexity further by making use of local
expansions. There are analytic expressions to shift the multipole expansion
coefficients $M_n^m$ to local expansion coefficients $L_n^m$ \footnote{see
appendix \ref{app:3d_laplace} for expressions for these shift operators in three
dimensions.}


\textbf{Notes from \cite{Ying:2004:JCP}}

FMM makes use of these representations in a recursive algorithm. computational
domain is a box containing all particles, sources and targets. Hierarchically
partitioned into a tree structure, called Octree in 3D. With each level $l$ of the
tree partitioned into $8^l$ geometric boxes. For each box, the potential induced
by it's source densities is represented by a multipole expansion centered around
the box, while the potential induced by the sources from non-adjacent boxes is
encoded in a local expansion.

Number of expansion terms $p$ is chosen for a prescribed relative error $\epsilon$,
using $p=\log_c \epsilon$ where $c=\frac{4-\sqrt{3}}{\sqrt{3}}$ in 3D (numerically optimal?).
The trucation error has rigorous bounds.

The availability of analytic translations enable the $\mathcal{O}(n)$ algorithm.
In particular the following translations; M2M, L2L, M2L.

Two basic steps of FMM using tree structure.

\begin{enumerate}
    \item \textit{Upward Pass} The tree is traversed, post-order. S2M at leaves
    - compute multipole expansion of leaf sources at leaf node. Shift multipole
    expansion to parent node, and sum together.
    \item \textit{Downward Pass} The tree is traversed pre-order. The local expansion
    for each box is the sum of two parts: (1) the L2L transformation collects the local
    expansion of B's parent (compressing the information for boxes non-adjacent to B's parent)
    (2) the M2l translation, for each multipole expansion of boxes which are the children of
    the neighbors of B's parent but not adjacent to B itself.At the end of the
    Downward Pass, the `far' interaction which is evaluated
    using the local expansion at this box (L2T operation) at each target particle.
    Combined with the `near' interaction by direct computation of potential over all
    source points in near field (within the box itself, and it's direct neighbors).
\end{enumerate}


\begin{figure}[!h]
    \centering
    {\includegraphics[width=1.1\textwidth]{introduction/main_loop.png}}
  \caption{foo bar}
  \label{fig:1_1_main_loop}
\end{figure}


\subsection{Analysis}

Begin with $O(N\log(N))$ algorithm variant; here post order traversal (from
coarsest to finest level), compute multipole expansion for each box at each level,
this has the $Np^2$ bound, at finest level assume $O(1)$ number of particles, so
direct computation with nearest neighbour particles leads to $O(N)$ bound.

Full analysis defer to \cite{Greengard:1987:Yale}. Enough to understand that the
translation operators are what lead to the $\mathcal{O}(n)$ complexity. Beginning
with upward pass, at leaf level each particle contributes to one expansion so
S2M of the order $Np^2$ - $p^2$ is the order of operations for the multipole
expansion, can see this from the equation. M2M/L2L/M2L shifts require $p^4$
operations, so all are bounded by $O(Np^4)$ (consider last level), precise bound
dictated by number of boxes in interaction list. Evaluating the $p^th$ degree
local expansions at each target particle (L2T) bounded by $O(Np^2)$, small constant
$\kappa$ particles enforced at leaf level leads to $O(\kappa N)$ complexity for
direct calculations at leaf level. We see that the whole algorithm is bounded by
$O(N)$.

\hspace{10pt}
