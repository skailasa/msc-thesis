\subsection{Motivation}

\textbf{Notes from \cite{Cipra:2000:SN}}

Accurate N-Body simulations (stars in a galaxy, atoms in a protein) require
$\mathcal{O}(n^2)$ calculations. Using multipole expansions in the far field,
and hierarchical partitioning of the space. Comes equipped with rigorous error bounds.

\hspace{10pt}

\textbf{Notes from \cite{Greengard:1987:Yale}}

\begin{itemize}
    \item particle simulation important in many fields: e.g. celestial mechanics
    \item 2 approaches, (1) solving N2L with given initial conditions and find trajectories as
    function of time
    \item or, (2) find equilibrium configuration by solving for potential.
\end{itemize}

Potential of the form

\begin{flalign}
    \Phi = \Phi_{\text{near}} + \Phi_{\text{external}} + \Phi_{\text{far}}
\end{flalign}

\begin{enumerate}
    \item $\Phi_{\text{external}}$, underlying potential (background).
    \item $\Phi_{\text{near}}$, rapid decaying (e.g. Van Der Waals)
    \item $\Phi_{\text{far}}$, far field, e.g Coulombic, Gravitational.
\end{enumerate}

1 and 2 are $\mathcal{O}(n)$ to evaluate at N points. 3 decays slower,
so all particles must be accounted for, naively evaluating results
in $\mathcal{O}(n^2)$

Shows up in celestial mechanics, plasma physics, molecular dynamics etc

\hspace{10pt}


\textbf{Notes from \cite{Beatson:1997:SCF}}

Again, reference efficacy in particle simulations for computing pairwise interactions.
As this is where they are most easily understood. Equally useful in the solution
of certain partial differential equations by first recasting as integral equations.

`Atomic' problem of many probs in comp. phys. boils down to,

Examples of physical problems,

Gravitational potential 3D

\begin{flalign}
    \phi(x_j) = \sum_{i = 1, \> i \neq j}^N\frac{m_i}{r_{i,j}}
\end{flalign}

Gravitational field 3D.

\begin{flalign}
    \mathbf{E}(\mathbf{x_j}) = \sum_{i = 1, \> i \neq j}^Nm_i \frac{\bf{x_j}-\bf{x_i}}{r_{i,j}^3}
\end{flalign}

Same as electrostatics, where coulomb's law used instead, mass $\rightarrow$ charge.

Can turn into integral, for continuous distributions of mass/charge. (Biot-Savart law etc)

Acoustic Scattering and diffusion, examples given too. Probably wont write in report.


Each physical problem results in integral equation,

\begin{flalign}
    u(\mathbf{x}) = \int K(\mathbf{x}, \mathbf{y}) w(\mathbf{y})d\mathbf{y}
\end{flalign}

Discrete case,

\begin{flalign}
    u(\mathbf{x}) = \sum_{i=1}^N w_i K(\mathbf{x}, \mathbf{y}_i)
\end{flalign}

Direct evaluations obviously quadratic computational cost. Most famous fast
summation method FFT $\sum_{k=1}^N e^{\frac{2\pi ijk}{N}}w_k$ in $\mathcal{O}(n \log(n))$.
FFT different, it's exact method, based on symmetry (algebra) - requires uniform spatial grid (geometry dep).
FMM are approximate - i.e. based on analytic considerations (vs algebra), insensitive
to source distribution (geometry).

FMM approach based on the following features

\begin{itemize}
    \item A specified acceptable accuracy $\epsilon$
    \item A hierarchical tree division of space
    \item a far field expansion of the kernel, in which the influence of the
    source and target points separates
    \item (optionally) the conversion of far field expansions into local expansions
\end{itemize}

\textbf{Idea}: Compress field in the far field as it's (relatively) smooth if it's well
separated from evaluation point. How? Using a `multipole expansion'. What are these?

\hspace{10pt}

\textbf{Notes from \cite{Griffiths:2017:CUP}}

page 152 develops the multipole expansion for the potential of an arbitrary
charge distribution in 3D.

\begin{flalign}
    V(\mathbf{r}) = \frac{1}{4 \pi \epsilon_0} \int \frac{1}{d}\rho(\mathbf{r}')d\tau'
\end{flalign}

$\mathbf{r}$ - vector from centre of expansion at charge distribution to the point
of evaluation, $d$ - the  distance from volume element $d\tau'$ to the point of
evaluation, $\rho(\mathbf{r}')$ - charge density, $\mathbf{r}'$ - the vector
from the centre of expansion to the volume element. $alpha$ is the angle between
$r$ and $r'$.

From law of the cosines,

\begin{flalign}
    d^2 &= r^2 + (r')^2 - 2rr'\cos \alpha = r^2 \left [ 1 + \left ( \frac{r'}{r} \right)^2 - 2 \left (\frac{r'}{r} \right)\cos \alpha \right]\\
    d &= r \sqrt{1+\epsilon}
\end{flalign}

Where,

\begin{flalign}
    \epsilon \equiv \left ( \frac{r'}{r} \right) \left (\frac{r'}{r} - 2 \cos \alpha \right)
\end{flalign}

$\epsilon$ is small far away from charge dist. so can expand $1/d$ binomially

\begin{flalign}
    \frac{1}{d} &= \frac{1}{r}(1+\epsilon)^{-1/2} = \frac{1}{r}\left (1 - \frac{1}{2}\epsilon + \frac{3}{8}\epsilon^2 - ... \right) \\
    \frac{1}{d} &= \frac{1}{r} \sum_{n=0}^{\infty} \left(\frac{r'}{r} \right)^n P_n(\cos \alpha)
\end{flalign}

So can write multipole expansion in terms of legendre polynomials coefficients
exactly in 3D,

\begin{flalign}
    V(\mathbf{r}) = \frac{1}{4 \pi \epsilon_0}\sum_{n=0}^{\infty}\frac{1}{r^{n+1}}\int (r')^nP_n(\cos \alpha)\rho(\mathbf{r'}) d \tau'
\end{flalign}

Again from \cite{Greengard:1987:Yale},

For discrete distributions of $N$ charges $q_i$ this goes to,

\begin{flalign}
    V(\mathbf{r}) = \frac{1}{4 \pi \epsilon_0}\sum_{i=1}^N\sum_{n=0}^{\infty}\frac{(r')^n q_i}{r^{n+1}}P_n(\cos \alpha)
\end{flalign}

Addition theorem for Legendre polynomials,

\begin{flalign}
    P_n(\cos \gamma) = \sum_{m=-n}^n Y_n^{-m}(\alpha, \beta) Y_n^m(\theta, \phi)
\end{flalign}

where two points P and Q with sph. coords. $(r, \theta, \phi)$ and $(\rho, \alpha, \beta)$
$\gamma$ is angle subtended between them,

From addition theorem, Multipole expansion goes to,

\begin{flalign}
    V(\mathbf{r}) &= \frac{1}{4 \pi \epsilon_0}\sum_{i=1}^N\sum_{n=0}^{\infty}\frac{(r')^n q_i}{r^{n+1}}P_n(\cos \alpha)\\
    & = \frac{1}{4 \pi \epsilon_0}\sum_{i=1}^N\sum_{n=0}^{\infty}\sum_{m=-n}^n \frac{(r')^n q_i Y_n^{-m}(\alpha_i, \beta_i) }{r^{n+1}}Y_n^m(\theta, \phi)\\
    & = \sum_{i=1}^N\sum_{n=0}^{\infty}\sum_{m=-n}^n\frac{M_n^m}{r^{n+1}} \cdot Y_n^m(\theta, \phi)
\end{flalign}

This is an exact expansion, can be truncated as reqiured for tunable precision.

If instead expansion taken with centre at targets, at distance from sources, Can
rewrite as Local Expansion.

\begin{flalign}
    V(\mathbf{r}) = \frac{1}{4 \pi \epsilon_0} \sum_{i=1}^N \sum_{j=0}^{\infty} \sum_{k=-j}^j L_j^k \cdot  Y_j^k(\theta, \phi) \cdot r^j
\end{flalign}

Where the coefficients of the local expansion can be derived from the coefficients
of the the multipole expansion.

Analytic expressions for shifting multipole and local expansions explained. Should
probably specify shift operators in appendix for completeness...

\hspace{10pt}

\subsection{Algorithm structure}

\textbf{Notes from \cite{Ying:2004:JCP}}

FMM makes use of these representations in a recursive algorithm. computational
domain is a box containing all particles, sources and targets,


\subsection{Analysis}