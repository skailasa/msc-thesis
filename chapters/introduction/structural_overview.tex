The major contribution of the work underlying this thesis, is a well-tested and
extensible Python implementation of the \gls{KIFMM} algorithm, \gls{PyExaFMM}. The \gls{KIFMM}
algorithm is chosen for its relative simplicity in comparison to the \gls{FMM},
and the ability to trivially extend an implementation to study new kernels, and
apply the \gls{FMM} in new problem settings. The decision to use Python as the implementation language arises from the desire
to produce an \gls{KIFMM} implementation that sacrifices some computational
performance for developer productivity, via the usage of object oriented design
principles that ensure future extensibility, as well as testability.

The remainder of this thesis develops on the mathematical background introduced in Chapter \ref{chpt:1_introduction}.
Chapter \ref{chpt:2_strategy_for_practical_implementation}, granularly examines
the mathematical and software techniques taken to optimise the implementation of
\gls{PyExaFMM}. Specifically, the introduction of key bottlenecks in Section \ref{sec:2_1_bottleneck_analysis},
the efficient construction of trees in Section \ref{sec:2_2_efficient_trees}, the implementation
of multiprocessing to distribute and cache the computation of the \gls{KIFMM}
operators in Section \ref{sec:2_3_operator_caching}, the low-rank SVD based
approximations applied to compress the \gls{M2L} operators in Section \ref{sec:2_4_svd_compression},
and finally the software design choices that were influenced by the desire to
make \gls{PyExaFMM} extensibile and testable, while leaving room to optimise
performance in Section \ref{sec:2_5_software_design}. Chapter \ref{chpt:3} begins
by benchmarking the performance of \gls{PyExaFMM} in Section \ref{sec:3_1_benchmarking},
on the model problem, in addition to providing some quantitative analysis
for the speedup offered by optimisations already implemented by \gls{PyExaFMM}.
Chapter \ref{chpt:3} then proceeds to examine optimum parameters for the
low-rank SVD approximation of the \gls{M2L} operation in Section \ref{sec:3_2_svd_params}.
We conclude with a discussion of the wider research context of \gls{PyExaFMM},
as well as avenues for future study in Chapter \ref{chpt:conclusion}.
