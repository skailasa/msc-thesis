As mentioned in Section \ref{sec:2_3_operator_caching}, \gls{PyExaFMM} compresses
a concatenated version of the \gls{M2L} operator matrices using an \gls{SVD}.
The utility of using an SVD in comparison to an eigenvalue decomposition lies
in the fact that it is defined on all matrices, real and complex
\cite{Trefethen:1997:SIAM}. From equation (\ref{eq:2_3_concatenated_m2l}), the
concatenated M2L operators,

\begin{equation}
    A = \left [ A_1 | A_2 | ... | A_I \right]
    \label{eq:2_4_concatenated_m2l}
\end{equation}

where, as before, $\{A_i | i \in [1, 2, ..., I]\}$ are the M2L operator matrices
for a given source and target box pair and $I$ is the size of a given target box's
interaction list, can be written as a single matrix $A$, where generally
$A \in \mathbb{C}^{T, I \cdot S}$. Here $T$ refers to the number of quadrature
points on a given target box's surface, and $S$ refers to the number of quadrature
points on all the source boxes in its interaction list's surfaces. For non-square
matrices, it is always the case that either the rows or columns (whichever is
greater in number), are linearly dependent. Therefore,

\begin{equation}
    \text{rank}(A) \leq \min (T, I \cdot S)
\end{equation}

using this fact, it's therefore possible to write a lower-rank approximation of $A$.
However, an even lower rank approximation is possible using an SVD,

\begin{equation}
    A = U \Sigma V^*
\end{equation}

this can also be rephrased as a weighted sum of rank one matrices, or put simply
in terms of the products of the left and right singular vectors \cite{Trefethen:1997:SIAM},

\begin{equation}
    A = \sum_{j=1}^{r}\sigma_j u_j v_j^*
\end{equation}

where $r = \text{rank} (A)$, $\{\sigma_j | j \in [1, ..., r] \}$ are the singular
values and $\{u_j | j \in [1, ..., r] \} $ and $\{v_j | j \in [1, ..., r] \}$
are the left and right singular vectors respectively.

However, noting that singular values are generally arranged in weakly increasing
order in terms of magnitude, one can say that the $\tau^{th}$ partial sum where
$\tau \leq r$ captures as much `energy' of $A$ as possible,

\begin{equation}
    A_\tau = \sum_{j=1}^{\tau}\sigma_j u_j v_j^*
    \label{eq:2_4_sum_svd}
\end{equation}

here the `energy' of an operator is defined in terms of either a 2-norm or the
Frobenius norm as developed in \cite{Trefethen:1997:SIAM}. In fact it can be shown
that for any $\tau$ with $0 \leq \tau \leq r$, if $\tau = \min \{T, I \cdot S\}$
and we define $\sigma_{\tau + 1} = 0$ then,

\begin{equation}
    ||A - A_\tau ||_2 = \sigma_{\tau+1} = 0
\end{equation}

meaning that $A_\tau$ is the best approximation of $A$, by a matrix of a lower rank
\cite{Trefethen:1997:SIAM}. However, one can also cut off the sum (\ref{eq:2_4_sum_svd}),
if the energy of the approximated $A_\tau$ is within some acceptable tolerance,

\begin{equation}
    ||A - A_\tau ||_2  \leq  \text{tol}
    \label{eq:2_4_svd_tol}
\end{equation}

such that if the sum is cut off at the $k^{th}$ value, leaving the approximation with
$\text{rank}(A_\tau) = k$, where $\sigma_k > \text{tol}$ and
$\sigma_{k+1} \leq \text{tol}$. The \gls{SVD} of the approximation $A_\tau$ can
then be written as,

\begin{equation}
    A_\tau = U_k \Sigma_k V_k^*
    \label{eq:2_4_decomposed_approximation_a_tau}
\end{equation}

where $U_k$ and $V_k$ have $k$ rows and columns respectively. As $k < r$, this
method is known as a low-rank SVD approximation. The application of
(\ref{eq:2_4_decomposed_approximation_a_tau}) represents a complexity saving in
comparison to applying $A$ directly. This can be seen by considering the full
SVD of a generic matrix $B \in \mathbb{C}^{m, n}$,

\begin{equation}
\begin{pNiceMatrix}[first-row,last-row,first-col,last-col]
 &    &    &   \leftarrow n \rightarrow  &     & \\
\uparrow &    &    &   &    &   & \\
m &    &    & B &    &  &  \\
\downarrow &    &    &   &    &  &  \\
 &    &    &   &    &      \\
\end{pNiceMatrix} = \begin{pNiceMatrix}[first-row, last-row, first-col, last-col]
&  &  \leftarrow m \rightarrow  &   &  \\
\uparrow &  &  &   & \\
m &  & U  &   & \\
\downarrow &  &   &   & \\
&  &   &   & \\
\end{pNiceMatrix} \begin{pNiceMatrix}[first-row, last-row, first-col, last-col]
&  &  \leftarrow n \rightarrow  &   &  \\
\uparrow &  &  &   & \\
m &  & \Sigma  &   & \\
\downarrow &  &   &   & \\
&  &   &   & \\
\end{pNiceMatrix}\begin{pNiceMatrix}[first-row, last-row, first-col, last-col]
&  &  \leftarrow n \rightarrow  &   &  \\
\uparrow &  &  &   & \\
n &  & V^*  &   & \\
\downarrow &  &   &   & \\
&  &   &   & \\
\end{pNiceMatrix}
\label{eq:2_4_b_svd}
\end{equation}

where the dimension of each matrix in the \gls{SVD} decomposition has been
illustrated. For an approximated matrix $B_\tau$, with $\text{rank}(B_\tau) = k$,
we find by applying (\ref{eq:2_4_sum_svd}),

\begin{equation}
    \begin{pNiceMatrix}[first-row,last-row,first-col,last-col]
        &    &    &   \leftarrow n \rightarrow  &     & \\
       \uparrow &    &    &   &    &   & \\
       m &    &    & B_\tau &    &  &  \\
       \downarrow &    &    &   &    &  &  \\
        &    &    &   &    &      \\
       \end{pNiceMatrix} = \begin{pNiceMatrix}[first-row, last-row, first-col, last-col]
       &  &  \leftarrow k \rightarrow  &   &  \\
       \uparrow &  &  &   & \\
       m &  & U  &   & \\
       \downarrow &  &   &   & \\
       &  &   &   & \\
       \end{pNiceMatrix} \begin{pNiceMatrix}[first-row, last-row, first-col, last-col]
       &  &  \leftarrow k \rightarrow  &   &  \\
       \uparrow &  &  &   & \\
       k &  & \Sigma  &   & \\
       \downarrow &  &   &   & \\
       &  &   &   & \\
       \end{pNiceMatrix}\begin{pNiceMatrix}[first-row, last-row, first-col, last-col]
       &  &  \leftarrow n \rightarrow  &   &  \\
       \uparrow &  &  &   & \\
       k &  & V^*  &   & \\
       \downarrow &  &   &   & \\
       &  &   &   & \\
       \end{pNiceMatrix}
\label{eq:2_4_b_tau_svd}
\end{equation}

If we were to apply the \gls{SVD} decomposition of $B$ to a given vector
$x \in \mathbb{C}^{n}$, this would result in an asymptotic complexity of
$O(m^2 + n^2)$ from (\ref{eq:2_4_b_svd}). Compared with the application of the
\gls{SVD} decomposition of $B_\tau$, which results in an asymptotic complexity
of $O(k(m + n))$. As long as $k < \min (m, n)$, this represents a complexity
saving and therefore faster matrix-vector products with the approximation
$B_\tau$.

Currently, \gls{PyExaFMM} computes the \gls{SVD} using the provided function
from the SciPy module, which itself calls a fast \textbf{\gls{LAPACK}}
function. However, this operation is optimised to reduce the number
of \textbf{\gls{FLOPS}}, rather than take advantage of advances in modern
\textbf{\gls{heterogenous}} and multicore machines, and
distributed memory programming paradigms which are more often limited by the
communication and data transfer overhead between processes, than by number of
operations \cite{Halko:2011:SIAM}. Halko et. al introduce new `randomised'
approaches for generating low-rank approximations of matrices, which can be
optimised for modern heterogenous and distributed computing environments as they
consist of processes which can be easily parallelised.

We roughly derive this method here in application to the \gls{SVD}, however guide
the reader to the literature for further detail \cite{Erichson:2019:JOSS,Halko:2011:SIAM}.
Consider again the matrix $B \in \mathbb{C}^{m, n}$. Randomised methods consist
of two logical steps to find low-rank approximations of $B$: Step one, construct
a low-dimensional subspace that approximates the column space of $B$, i.e.
find an orthonormal matrix $Q \in \mathbb{C}^{m, k}$ such that,

\begin{equation}
    B \approx QQ^*B
    \label{eq:2_4_step_1_randomised}
\end{equation}

where $k$ takes the same meaning as before as the target rank of the compressed
matrix. Step two, form a smaller matrix defined $C := Q^*B \> \in \mathbb{C}^{k, n}$,
by which $B$ is restricted to a lower dimensional space spanned by the basis
$Q$.

Step one is `randomised' by drawing $k$ random vectors $\{ \omega_i \}_{i=1}^k$,
where the elements are drawn from a normal distribution, and finding the random
resulting projections due to $B$, $y_i = B \omega_i$. In matrix form we can write,

\begin{equation}
    Y := B \Omega
\end{equation}

where $\Omega$ is a matrix with columns formed formed from $\omega_i$.
Probability theory guarantees a high-chance of each $y_i$ being linearly independent
\cite{Erichson:2019:JOSS}. This matrix can be orthonormalised via a QR decomposition
to find,

\begin{equation}
    Y =: QR
\end{equation}

where $Q$ is the orthonormal basis we desire, and $R$ as usual is an upper triangular
matrix. This definition of $Q$ satisfies (\ref{eq:2_4_step_1_randomised}). Step two
is now computed as,

\begin{equation}
    C := Q^* B
    \label{eq:2_4_step_2_randomised}
\end{equation}

which provides the compressed matrix $C \in \mathbb{C}^{k, n}$.

The low-rank approximation implemented in \gls{PyExaFMM} via a truncated
\gls{SVD} (\ref{eq:2_4_decomposed_approximation_a_tau}) is costly to compute
for large matrices. This is because a full \gls{SVD} decomposition must be computed, of
which all but the first $k$ terms are ignored. However, we can now embed the
\gls{SVD} into the randomised framework above. Consider a compressed matrix
$C \in \mathbb{C}^{k, n}$ obtained via the two steps detailed above. An
ordinary deterministic implementation can be used to calculate the
the \gls{SVD} of $C$ cheaply in comparison to the full SVD of $B$ as long as
$k \ll n$, such that

\begin{equation}
    C = \tilde{U}\Sigma V^*
    \label{eq:2_4_svd_of_c}
\end{equation}

here we cheaply obtain the first $k$ right singular vectors of $B$,
from $V \in \mathbb{C}^{n, k}$, as well as the first $k$ singular values
$\Sigma \in \mathbb{R}^{k, k}$. To find the entire \gls{SVD} of $B$ we notice
that,

\begin{equation}
    B \approx Q Q^* B = QC = Q \tilde{U} \Sigma V^* := U\Sigma V^*
\end{equation}

where we combine (\ref{eq:2_4_step_1_randomised}) and (\ref{eq:2_4_step_2_randomised})
with the results of the \gls{SVD} of $C$ (\ref{eq:2_4_svd_of_c}), and define the
left singular vectors as $U = Q \tilde{U}$. In terms of \gls{PyExaFMM}, randomised
methods for low-rank approximations are attractive candidates for future implementation
as they represent the state of the art for accelerating the computation of an \gls{SVD}.
As noted in Chapter \ref{chpt:1_introduction} Section \ref{sec:1_2_kifmm_overview},
\gls{PyExaFMM}'s approach differs to other major \gls{KIFMM} implementations \cite{Malhotra:2015:CCP,exafmm}.
The implementation of randomised low-rank SVD approximations for accelerating the calculation of the \gls{M2L}
operator matrices will therefore make an interesting point of comparison. Furthermore,
an open source \textbf{\gls{apache}} \textbf{software foundation} lead software implementation for
a stocastic \gls{SVD} based on these methods is readily available \cite{mahout}.
