Some of the key bottlenecks in implementing the \gls{FMM} and \gls{KIFMM}, and potential
optimisation strategies in the computation of these algorithms, have already been
discussed. This chapter adds detail to the approaches used by \gls{PyExaFMM} to tackle
the issues raised. Namely, the approach to multiprocessing and caching the \gls{M2M},
\gls{L2L} and \gls{M2L} operations, and the acceleration of the \gls{M2L} calculation
via a low-rank approximation using an \gls{SVD}. The other significant bottlenecks
and implementation issues addressed thus far in the implementation of \gls{PyExaFMM} are the
efficient construction of tree data structures, and software design considerations.

\gls{PyExaFMM}'s design philosophy is based on the twin goals of testability
and extensibility. The continuous nature of the results of numerical codes makes
debugging uniquely challenging, as the nature of the bug could be a numerical error,
an error in implementation of algorithm logic, or an error due to the misuse of a
programming language or frameworks. Testability for \gls{PyExaFMM} refers to
the attempt to create stable interfaces around algorithmic logic, return values,
and data input and output. Furthermore, as the current implementation represents
in many ways the first iteration of the software, we adopt object-oriented
design principles to ensure that new optimisations and extensions are easy
to implement in the future. Using Python, an \textbf{\gls{interpreted}} and
\textbf{\gls{object-oriented-language}}, also results in a tradeoff in terms of
computational performance for developer productivity - which further impacts
on design preferences, and pushes \gls{PyExaFMM} to prefer a small number of
simple objects, without relying too much on inheritance as many optimisation tools
rely on access to underlying data containers.

As with all optimised numeric codes, \gls{PyExaFMM} relies on an efficient
vectorised data structures, specifically an efficient representation of the tree
data structure used in the main \gls{FMM} loop. This is done via an Morton encoding,
which has the effect of translating multidimensional data to one dimension whilst
preserving the data's locality. This allows for an efficient vector representation
of box centers, at each level of a tree, and for the application of optimised
numeric libraries.
