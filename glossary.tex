\newglossaryentry{latex}
{
    name=latex,
    description={Is a mark up language specially suited
    for scientific documents}
}

\newglossaryentry{far-field}{
    name={far field},
    description={
        An set of particles in the far field are considered to be far away enough
        from the particle of interest that they are suitably described by a
        convergent multipole expansion.
    }
}

\newglossaryentry{near-field}{
    name={near field},
    description={
        An set of particles in the near-field are considered to violate the
        the criteria for describing them with a convergence multipole expansion.
    }
}

\newglossaryentry{near-neighbours}
{
    name={near neighbours},
    description={
        Two boxes in computational tree are near neighbours if they are at the
        same level of refinement, and share a boundary point.
    }
}

\newglossaryentry{well-separated}
{
    name={well separated},
    description={
        Two boxes in computational tree are near neighbours if they are at the
        same level of refinement, and are not near neighbours.
    }
}

\newglossaryentry{interaction-list}{
    name={interaction list},
    description={
    Boxes which are the children of the near-neighbours of the a box's
    parent box, but are not adjacent to the box itself.
    }
}

\newglossaryentry{FMM}{
    name={FMM},
    description={
        The Fast Multipole Method.
    }
}

\newglossaryentry{KIFMM}{
    name={KIFMM},
    description={
        The `Kernel Independent' Fast Multipole Method.
    }
}

\newglossaryentry{post-order}{
    name={post-order},
    description={
        In reference to the traversal of heirarchical trees, post-order traversal
        is moving from the finest (leaf) level to the coarsest (root) level.
    }
}

\newglossaryentry{pre-order}{
    name={pre-order},
    description={
        In reference to the traversal of heirarchical trees, pre-order traversal
        is moving from the coarsest (root) level to the finest (leaf) level.
    }
}

\newglossaryentry{P2M}{
    name={P2M},
    description={Particle to multipole expasion operation.}
}

\newglossaryentry{M2M}{
    name={M2M},
    description={Translate from multipole of child box to multipole expansion
    of parent box.}
}

\newglossaryentry{M2L}{
    name={M2L},
    description={Translate from multipole of box $A$ to local expansion
    of box $B$.}
}

\newglossaryentry{L2L}{
    name={L2L},
    description={Translate from local of parent box to local expansion
    of child box.}
}

\newglossaryentry{L2P}{
    name={L2P},
    description={Evaluate local of leaf box at each target particle in a leaf
    box.}
}

\newglossaryentry{P2P}{
    name={P2P},
    description={
        Evaluate particle to particle interactions directly.
    }
}

\newglossaryentry{target-particles}{
    name={target particles},
    description={
        Particles are described as targets if the potential due to a source field
        is being evaluated at these particle's positions, but they themselves are
        not being considered as a source for the field being evaluated. They may
        or may not refer to the same set of particles as the source particles,
        for example in the classic $N$-Body problem, the target particles and
        the source particles are the same set.
    }
}

\newglossaryentry{source-particles}{
    name={source particles},
    description={
        Particles are described as sources if they are consifered to contribute
        to the source field being evaluated at the target particles. They may
        or may not refer to the same set of particles as the target particles,
        for example in the classic $N$-Body problem, the target particles and
        the source particles are the same set.
    }
}

\newglossaryentry{task-level-parallelism}{
    name={task-level parallelism},
    description={
        Task-level parallelism is achieved when multiple threads or processes,
        running the same, or differing, code, in a multiprocessing system are
        executed with the same, or differing, data.
    }
}

\newglossaryentry{instruction-level-parallelism}{
    name={instruction-level-parallelism},
    description={
        Instruction-level parallelism refers to the parallel execution of different
        instructions on a specific thread or process in a multiprocessing system.
    }
}

\newglossaryentry{data-level-parallelism}{
    name={data-level parallelism},
    description={
        In a multiprocessor system, data level parallelism is achieved by
        distributing data amongst different compute nodes to be executed upon
        in parallel. See \textbf{CUDA}
    }
}

\newglossaryentry{CUDA}{
    name={CUDA},
    description={
        `Compute Unified Device Architecture', is a parallel computing API
        designed for the development of programs for \textbf{GPU}s
    }
}

\newglossaryentry{GPU}{
    name={GPU},
    description={
        `Graphics Processing Unit', are specialised processors specifically
        designed for rapid parallel execution across large blocks of data.
    }
}

\newglossaryentry{SIMD}{
    name={SIMD},
    description={
        Single Instruction, Multiple Data.
    }
}

\newglossaryentry{high-level-interpreted-language}{
    name={high level interpreted language},
    description={
        A High-Level language is one that offers an API and primitive data
        structures that strongly abstract from the details of the computer on
        which it is being run. An interpreted language is one in which codes
        are run directly via an \textit{interpreter}, rather than first being
        compiled. In reality interpreted languages run in a `virtual machine',
        which takes input code, transforms it to byte-code which is then translated
        into machine level instructions. This approach allows for greater portability
        of code, and developer productivity, at the expense of space and time overhead
        in running software.
    }
}

\newglossaryentry{source-densities}{
    name={source densities},
    description={
        Source densities are associated with their respective \textbf{source particles}.
        In electromagnetic problems, these would correspond to charges. In gravitational
        problems, these would correspond to mass.
    }
}

\newglossaryentry{equivalent-density}{
    name={equivalent density},
    description={
        An equivalent representation of the potential generated by a discrete/continuous
        distribution. This equivalent density is supported at discrete points on
        an \textbf{equivalent surface}.
    }
}

\newglossaryentry{equivalent-surface}{
    name={equivalent surface},
    description={
        An equivalent surface supports equivalent density points.
    }
}

\newglossaryentry{check-surface}{
    name={check surface},
    description={
        Defines the surface at which the potential caused by source points, and
        the equivalent density formulation coincide.
    }
}

\newglossaryentry{check-potential}{
    name={check potential},
    description={
        The potential calculated from source particles, or an equivalent density
        distribution at a check surface.
    }
}

\newglossaryentry{PDE}{
    name={PDE},
    description={
        Partial Differential Equation.
    }
}

\newglossaryentry{SVD}{
    name={SVD},
    description={
        Singular Value Decomposition.
    }
}

\newglossaryentry{FFT}{
    name={FFT},
    description={
        Fast Fourier Transform.
    }
}

\newglossaryentry{MPI}{
    name={MPI},
    description={
        Message Passing Interface for implementing distributed memory parallelism.
        Each process is given it's own local memory, and can communicate and pass
        data and computational results to other processes being run in parallel.
    }
}

\newglossaryentry{OpenMP}{
    name={OpenMP},
    description={
        Open MultiProcessing is an API for implementing shared-memory parallelism.
        Each process is run on a separate thread, but a global memory space is
        shared.
    }
}

\newglossaryentry{PyExaFMM}{
    name={PyExaFMM},
    description={
        The Python implementation of the \textbf{KIFMM} presented as a part of this
        thesis.
    }
}

\newglossaryentry{object-oriented-language}{
    name={Object Oriented Language},
    description={
        Programming paradigm in which data and methods are abstracted into
        `objects'. This allows for the separation of data and logic into
        structures that encapsulate their logical grouping.
    }
}

\newglossaryentry{interpreted}{
    name={interpreted},
    description={
        In reference to an interpreted programming language. Here instead of
        direct compilation of source code into machine level instructions, source
        code is passed to an `interpreter', which compiles a byte-code on the programmers
        behalf. This byte-code is sent to a compiler as it arrives, giving the programmer
        the illusion of interactive programming.
    }
}

\newglossaryentry{JIT}{
    name={JIT},
    description={
        Just-In-Time compilation ...
    }
}
