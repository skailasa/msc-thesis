\chapter{Conclusion}\label{chpt:conclusion}

In summary, the main contribution of the work underlying this thesis is
\gls{PyExaFMM}, which represents the first iteration of a three dimensional
\gls{KIFMM} simulation library, with some parallel features. The software has been
designed to be well testable and extensible, however is currently someway behind
state-of-the art implementations in terms of performance
\cite{exafmm,Malhotra:2015:CCP}.

generic functionality

- implementation of gradients for benchmarking with open source standards

- implementation of new kernels.

trees

- Parallel tree construction

- An adaptive algorithm

software design

- more integration tests due to complexity.

- reduction in boiler plate code for data loading etc, in terms of design.

- new module for data handling.

compression

- More robust SVD compression, randomized compression.

- Taking more check points than multipole/local points for more robust pseudo-inverse.

parallelism

- Offloading M2L computation to specialized hardware
